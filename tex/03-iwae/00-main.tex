\section{Implementing the Importance Weighted Autoencoder (IWAE)}

While the ELBO serves as a lower bound to the true marginal log-likelihood, it may be loose if the variational 
posterior $q_{\phi}(\bz \mid \bx)$ is a poor approximation to the true posterior $p_{\theta}(\bz \mid \bx)$. 
It is worth noting that, for a fixed choice of $\bx$, the ELBO is, in expectation, the log of the unnormalized density ratio:

\begin{equation}\label{eq:12}
    \frac{p_{\theta}(\bx, \bz)}{q_{\phi}(\bz \mid \bx)} = \frac{p_{\theta}(\bz \mid \bx)}{q_{\phi}(\bz \mid \bx)} \cdot p_{\theta}(\bx)
\end{equation}

where $\bz \sim q_{\phi}(\bz \mid \bx)$. The term $\frac{p_{\theta}(\bz \mid \bx)}{q_{\phi}(\bz \mid \bx)}$ is normalized such that

\begin{equation}
    E_{q_{\phi}(\bz \mid \bx)}[\frac{p_{\theta}(\bz \mid \bx)}{q_{\phi}(\bz \mid \bx)}] = \int q_{\phi}(\bz \mid \bx)\frac{p_{\theta}(\bz \mid \bx)}{q_{\phi}(\bz \mid \bx)} d \bz = \int p_{\theta}(\bz \mid \bx) d \bz = 1.
\end{equation}

meaning that if you add the $p_{\theta}(\bx)$ term, it no longer integrates to 1. 

As can be seen from the RHS, the density ratio is \textit{unnormalized} since the 
density ratio is multiplied by the constant $p_{\theta}(\bx)$. We can obtain a tighter bound by averaging multiple 
unnormalized density ratios. This is the key idea behind IWAE, which uses $m > 1$ samples from the approximate posterior 
$q_{\phi}(\bz \mid \bx)$ to obtain the following IWAE bound:

\begin{equation}\label{eq:13}
    \calL_{m}(\bx;\theta,\phi) = \E_{\bz^{(1)},...,\bz^{(m)} \overset{\text{i.i.d.}}{\sim} q_{\phi}(\bz \mid \bx)} \left(\log \frac{1}{m} \sum\limits_{i=1}^{m} \frac{p_{\theta}\left(\bx, \bz^{(i)}\right)}{q_{\phi}\left(\bz^{(i)} \mid \bx\right)}\right)
\end{equation}

Notice that for the special case of $m = 1$, the IWAE objective $\calL_m$  reduces to the standard ELBO $L_1 = \E_{z\sim q_{\phi}(\bz \mid \bx)}\left(\log \frac{p_{\theta}\left(\bx, \bz\right)}{q_{\phi}\left(\bz \mid \bx\right)}\right)$.

\begin{enumerate}[label=(\alph*)]
    \item \input{03-iwae/01-iwae-lowerbound}

    \item \points{3b} Implement IWAE for VAE in the \texttt{negative\_iwae\_bound} function in \texttt{vae.py}. The functions \texttt{duplicate}
    and \texttt{log\_mean\_exp} defined in \texttt{utils.py} will be helpful.

    \item \points{3c} Run the command below to evaluate your implementation against the test subset:
    \begin{verbatim}
        python main.py --model vae --iwae
    \end{verbatim}

    This will output IWAE bounds for $m = \{1,10,100,1000\}$. Check that the IWAE-1 result is consistent with 
    your reported ELBO for the VAE. All four IWAE bounds for VAE will be reported to:
    \begin{enumerate}
        \item \texttt{submission/VAE\_iwae\_1.pkl}
        \item \texttt{submission/VAE\_iwae\_10.pkl}
        \item \texttt{submission/VAE\_iwae\_100.pkl}
        \item \texttt{submission/VAE\_iwae\_1000.pkl}
    \end{enumerate}

    To check the accuracy of your results, run:
    \begin{verbatim}
        python grader.py 3c-0-basic
        python grader.py 3c-1-basic
        python grader.py 3c-2-basic
        python grader.py 3c-3-basic
    \end{verbatim}

    \item \points{3d} As IWAE only requires the averaging of multiple unnormalized density ratios, the IWAE bound is also applicable 
    to the GMVAE model. Repeat parts 2 and 3 for the GMVAE by implementing the \texttt{negative\_iwae\_bound} function in 
    \texttt{gmvae.py}. 
    
    Run 
    \begin{verbatim}
        python main.py --model gmvae --iwae
    \end{verbatim}
   to evaluate your implementation against the test subset. 
   
   All four IWAE bounds for GMVAE will be reported to:

   \begin{enumerate}
        \item \texttt{submission/GMVAE\_iwae\_1.pkl}
        \item \texttt{submission/GMVAE\_iwae\_10.pkl}
        \item \texttt{submission/GMVAE\_iwae\_100.pkl}
        \item \texttt{submission/GMVAE\_iwae\_1000.pkl}
    \end{enumerate}

    To check the accuracy of your results, run:
    \begin{verbatim}
        python grader.py 3d-0-basic
        python grader.py 3d-1-basic
        python grader.py 3d-2-basic
        python grader.py 3d-3-basic
    \end{verbatim}

\end{enumerate}